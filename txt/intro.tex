\paragraph{Motivation: The economics of educational services}

"The Bottom Of the Pyramid" (BOP) is a term central to the development of
value-chains in emerging markets in developing countries. It is understood as
the largest, although poorest segment of the population. Numerous works, in
particular Coimbatore Krishnarao Prahalad's seminal work "The Fortune at the
Bottom of the Pyramid", suggests that the key to economic development then lies
in activating this segment of the population, enabling them not only to use
modern (digital) services, but also author them as 
well\cite{prahalad2009fortune}.

The inspiration for this idea comes from applying this general framework to the
field of education software. Alas, most software is written for only a small
subset of the population. H5P\footnote{\url{http://h5p.org/}} is one of many 
novel examples of software providing added-value to tech-savvy teachers and 
educational software developers. However, the BOP would in this case be the 
students.

By providing the correct tooling and educational framework, we enable students
to act not only as users of off-the-shelf (OTS) educational software, but
also as authors of their own or their co-students' learning experience. By
doing so, we allow the student to feel some degree of ownership to the
coursework, allowing for reinforced learning, as well as sharing (viral) effects
(perhaps due to pride), a major force in the modern network economy.

Such a framework would also encourage the students to create services and tools
more adapted to their particular learning situation, allowing for an overall
better user experience, as well as a valuable source of inspiration (and usage
data/analytics) for educational software developers and user interface
designers.
