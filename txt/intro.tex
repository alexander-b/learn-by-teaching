\paragraph{Motivation: The economics of educational services}

"The Bottom Of the Pyramid" (BOP) is a term central to the development of
value-chains in emerging markets in developing countries. It is understood as
the largest, although poorest segment of the population. Numerous works, in
particular Coimbatore Krishnarao Prahalad's seminal work "The Fortune at the
Bottom of the Pyramid", suggests that the key to economic development then lies
in activating this segment of the population, enabling them not only to use
modern (digital) services, but to author them as 
well\cite{prahalad2009fortune}.

The inspiration for our idea comes from applying this general framework to the
field of education software.
