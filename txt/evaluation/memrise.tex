\subsection{Evaluation of Memrise}

Memrise is a flashcard-based learning system. The cards make use of (often
crowdsourced) mnenonics combined with spaced repetition (akin to Duolingo and
Anki) to augment the learning experience. It is unique in part due to its
openness regarding its techniques and its close ties with academic researchers,
who are allowed to analysis its datasets.

Memrise itself suggests three basic techniques:\cite{memrise}

\begin{itemize}

\item textbf{Elaborate encoding.} That is, the construction of cognitive structures of
  already known material that new knowledge might "attach" itself to. For
  example, taxonomies (like the colours red, white, blue) are easier to remember
  than a random list (astronaut, velvet, cigar). To this end, Memrise uses
  crowdsourcing (with machine-learning techniques) to suggest "mems", which are
  the aforementioned cognitive structures to boost learning.

\item \textbf{Choreographed testing}, which test recall and comprehension. Memrise varies
  between simple one-off challenges ("casa = ?") and multiple-choice questions
  to keep things interesting.

\item \textbf{Scheduled reminders}, which is basically spaced learning. As
  Memrise themselves state: "Research suggests that reminders are most effective
  when they occur just before a memory fades completely and that successive
  reminders should be separated by longer and longer intervals. "\cite{memrise}

\end{itemize}

Research around or by Memrise mostly revolves around the benefits of testing,
error generation (vis-a-vis errorless learning), and how spacing affects the
efficacy of the learning programme. In short:

\begin{itemize}

\item Testing is the act of challenging retention/comprehension while studying,
  which is benefitial as part of the learning process. There is no consensus as
  to why it works, but some explanations include a) it increases the storage
  strength of memory, b) it generates additional cues that creates (potentially
  more efficient) routes through memory.\cite[p.6]{potts2014benefit}

\item Generation (with feedback) is an active form of learning, and requires the
  student to complete something, for example a sentence ("Lisa is a \_\_\_"),
  before getting to know the answer. This is contrasted to reading, where the
  student absorbs textual material and definitions, and multiple choice, which
  amongst other things tests the ability of the student to see how the different
  alternatives relate to each other (finding the odd man out etc).

\item Errorful learning has been thought to be detrimental to learning as people
  recall their errors to a greater extent than their correct answers (pp. 6),
  but this research has mostly been done with already memory-impaired
  populations, which might have different requirements. Errorless learning does
  not seem to be as effective as error generation: "... generating responses
  followed by feedback is helpful to memory even when many errors are generated,
  compared with errorless studying without
  generation."\cite[p.54]{potts2014benefit}

\item "In conclusion, generating errors benefitted vocabulary learning even when
  the items to be learned had no pre-existing associations, but participants did
  not predict this benefit"\cite[p.54]{potts2014benefit}. Likewise: "Of the
  different kinds of generation rules used, completing sentences appears to lead
  to the biggest advantage over reading only, but even relatively minor types of
  generation (like switching two underlined letters) improves test scores over
  reading alone"\cite[p.73]{benassi2014applying}.

\end{itemize}

Several experiments have been designed around to test assumptions regarding
these models. They are mostly language-oriented (a good fit for a programming
language!), and might use techniques like archaic words (which are surely
unknown for most test subjects) to provide a "clean slate" for testing
comprehension and retention of definitions. Three such experiments are provided
in \cite{potts2014benefit}.

\subsubsection{Expressiveness}

More or less any knowledge that needs to be internalised can fit within the
style of testing offered by Memrise. Users are allowed to create their own
courses, which may contain multimedia levels. For example, users may embed Vines
inside mems to boost retention. 

\subsubsection{Ease of use}

Simple interface with a few, mostly obvious buttons. Flow is never a problem.
Interactive modals are provided and guide the user through things like creating
mems for their own courses and so on.

\subsubsection{Takeaway points}

\begin{itemize}
\item Generating (that is, having the student do something) is provably more
  efficient than either reading or quizzing.
\item We need to embrace generating errors as a way of learning, and give feedback
  as quickly as possible to promote retention and comprehension.
\end{itemize}

