\subsection{Evaluation of Minecraft}

Minecraft\footnote{\url{https://minecraft.net/}} is a modern open-world 
(sandbox) game. It allows the user to freely place, as well as craft 
(potentially new) blocks in the game. Thus, it is possible to build whatever 
imagination and time allows for in the game. The game is very popular, 
especially with the younger demographics, and has been noted for its ability 
to inspire creativity in its users\cite{minecraftign}.

Being a virtual world, and due to its popularity and game mechanics, it has been
suggested as a tool in primary and secondary education. Partly due to its 
potential for stimulating a sense of presence in the 
students\cite{coudrayminecraft}, which games in general are notoriously 
successful at.

In the area of situated learning, Minecraft tends to be classified as a
multi-user virtual environment, which is of particular interest to teachers
and students alike. But it is a category of situated learning which tends to 
suffer from some common issues, ranging from standardisation (can others 
extend it?) to time (how quickly is the user able to use the 
game?)\cite{dawley2014situated}. Due to its unique proposition of low cost, 
huge popularity, and innate creativity, Minecraft seems to offer a novel 
platform for learning to take place\cite{coudrayminecraft}.

\subsubsection{Expressiveness}

Minecraft is very expressive in allowing the user to roam freely, harvest 
resources, build structures, and craft new tools. As such, there are many 
modifications available to extend the game with new functionality, like energy 
systems, or new entities. There is currently no official application 
programming interface for developers of such extensions, and most of the 
extending is done using disassembly and patching the game locally.

Minecraft offers a primitive realisation of boolean digital circuits named
``redstone''. Redstone is a resource found in the game, and might in its 
harvested
state be placed on the ground to carry signals. Redstone carries either a
logical low (false) or high (true) signal, which can be (de)activated through 
a number of means, for example levers. Redstone can also be used in recipes to
craft new blocks like signal repeaters or switchable lights.

More notably, the redstone system in Minecraft is able to emulate the primitives
needed to construct any analogue circuit. Thus, people are able to construct
everything from simple systems that open doors, to full-on 8-bit
processors\cite{minecraftutube}. Thus, the redstone system offers an interesting
value-proposition: Mojang has succesfully fooled hundreds of thousands of 
children and adults alike into understanding primitive digital
circuits. By correspondence, they understand some set of boolean algebra, making
them capable programmers as well.

\subsubsection{Ease of use}

Minecraft is readily accessible, but has a steep learning 
curve\cite{minecraftign}. Systems like TooManyItems\cite{toomanyitems} makes 
the game slightly more accessible by making it easy to peruse recipes, aiding 
discovery. However, Minecraft itself provides little of a set path for playing 
the game, save a final boss fight, which is optional, and pretty much 
orthogonal to the ordinary game mechanics.

\subsubsection{Takeaway points}

\begin{itemize}
\item There are a number of issues with using virtual worlds in situated
  learning. Minecraft provides a reasonable solution to many of them.
\item Seemingly complicated topics (like circuits) can in fact be taught to
  children, given the right incentives 
\end{itemize}
