\subsection{Evaluation of Minecraft}

\textit{Minecraft} is a modern open-world (sandbox) game developed by Mojang. It
allows the user to freely place, as well as craft (potentially new) blocks in
the game. Thus, it is possible to build whatever imagination and time allows for
in the game. The game is very popular, especially with the younger demographics,
and has been noted for its ability to inspire creativity in its
users.\cite{minecraftign}

Being a virtual world, and due to its popularity and game mechanics, it has been
suggested as a tool in primary and secondary education. As noted by
\cite{coudrayminecraft}: `The topic of this study is to underline Minecraft's
potential for stimulating a sense of presence, out of, or at least before, any
learning activity, in an online context, i.e. to give the students this feeling
of 'being here' often reported to be missed in distance programmes`.

In the area of situated learning, Minecraft tends to be classified as a
multi-user virtual environment (MUVE), a set of particular interest to teachers
and students alike. This particular class of worlds tend to suffer from issues
ranging from standardisation (can others extend it?) to time (how quickly is the
user able to use the game?). A discussion of such issues is provided in
\cite{dawley2014situated}. \cite{coudrayminecraft} argues that, due to its
unique proposition of low cost, huge popularity, and innate creativity,
Minecraft seems to offer a novel platform for learning to take place.

\subsubsection{Expressiveness}

As already noted, Minecraft is very expressive in allowing the user to roam
freely, harvest resources, build structures, and craft new tools. As such, there
is a rich eco-system of mods extending Minecraft with new functionality, like
energy systems, or new entitites (`mobs`). It must be noted that there is
currently no official API for mod developers, and most of the extension is done
using disassembly and monkeypatching.

Minecraft offers a primitive realisation of boolean digital circuits named
`redstone`. Redstone is a resource found in the game, and might in its harvested
state be placed on the ground to carry signals. Redstone carries either a
logical low (false) or high (true) signal, which can be (de)activated by a
number of means, for example levers. Redstone can also be used in recipes to
craft new blocks like signal repeaters or switchable lights.

More notably, the redstone system in Minecraft is able to emulate the primitives
needed to construct any analogue circuit. Thus, people are able to construct
everything from simple systems that open doors, to full-on 8-bit
processors\cite{minecraftutube}. Thus, the redstone system offers an interesting
value-proposition: Mojang has succesfully fooled tens of thousands, if not
hundreds of children and adults alike into understanding primitive digital
circuits. By correspondence, they understand some set of boolean algebra, making
them into capable programmers as well.

\subsubsection{Ease of use}

As noted by \cite{minecraftign}, Minecraft is readily accessible, but with a
steep learning curve. Systems like TooManyItems\cite{toomanyitems} makes the
game slightly more accessible by making it easy to peruse recipes, aiding
discovery. However, Minecraft itself provides little of a set path for playing
the game (`on-rails mechanics`), save a final boss fight, which is optional, and
pretty much orthogonal to the ordinary game mechanics.

\subsubsection{Takeaway points}

\begin{itemize}
\item There are a number of issues with using virtual worlds in situated
  learning. Minecraft provides a reasonable solution to many of them.
\item Seemingly complicated topics (like circuits) can in fact be taught to
  children, given the right incentives 
\end{itemize}
