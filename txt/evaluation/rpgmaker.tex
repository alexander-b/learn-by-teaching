\begin{CJK}{UTF8}{min}
\subsection{Evaluation of RPGツクール}
\end{CJK}

For something different, let's consider 
\begin{CJK}{UTF8}{min}RPGツクール\end{CJK}. It's a game authoring tool wherein 
users create worlds through a tile-based map editor, and breathe life into 
their worlds by scripting actors, events and objects. Users may import their 
graphics and scripts, or use some of the wealth of art assets, and scripted 
events and characters that are distributed with the program. The series is 
primarily successful in Japan, with English translations being a rather new 
thing.

While the main reason for us considering 
\begin{CJK}{UTF8}{min}RPGツクール\end{CJK} is its engaging user interface, it 
has in fact been used successfully for teaching. Authoring games in 
\begin{CJK}{UTF8}{min}RPGツクール\end{CJK} has proven successful in teaching 
mathematics\cite{maltempi2004learning}, and games authored in 
\begin{CJK}{UTF8}{min}RPGツクール\end{CJK} have been used for encouraging 
programming students\cite{Ralph_1the}. It is no secret that authoring and 
playing games is a sound way of both encouraging learners and increasing their 
efficacy.

\begin{CJK}{UTF8}{min}RPGツクール\end{CJK} assumes little or no experience in 
programming or designing games, and can as such be said to be aimed at 
newbies. However, several very successful commercial computer games by 
experienced developers were authored using 
\begin{CJK}{UTF8}{min}RPGツクール\end{CJK}.

\subsubsection{Expressiveness}

\begin{CJK}{UTF8}{min}RPGツクール\end{CJK} is primarily aimed at making 
Japanese Role-Playing Games such as the old 
\begin{CJK}{UTF8}{min}ファイナルファンタジ\end{CJK} or 
\begin{CJK}{UTF8}{min}イース\end{CJK} games. The user may design tile-based 2D 
overhead-view maps with different types of graphics for the tiles, 
representing grass, ocean, and so on. Sprite sheets are used for animating 
characters. The user can script events and design a program flow through this.

Designing a map can in many ways be compared to using a bitmap editor. What 
You See Is What You Get. Authoring art assets outside of the ones that are 
distributed with the program is usually done with dedicated tools, such as 
using a video editor for making cut scenes, or an audio editor for making 
sound clips, or a bitmap editor for making sprite sheets. More recent versions 
have tools such as the character generator in which sprite sheets are actually 
generated by having the user choose settings such as the hair style and skin 
colour of the character they wish to generate.

The scripts for events and game objects in general is very powerful. The 
programming language Ruby is used in recent versions. Ruby is a fully-fledged 
Turing-complete programming language. The tool provides several high-level 
constructs to facilitate the scripting. Additionally there are tools for 
authoring scenarios and events through graphical user interfaces, comparable 
to the character generator.

\subsubsection{Ease of use}

The latest edition of \begin{CJK}{UTF8}{min}RPGツクール\end{CJK} boasts being 
``simple enough for a child; powerful enough for a 
developer''\cite{rpgmakervxace}. While certainly (theoretically) powerful, 
professional reviews are less than kind when discussing the user 
interface\cite{johnrpg} --- just like they've historically been unfavourable 
to the series\cite{gamespotrpg}.

However, interestingly, user reviews seems very positive on most 
sites\cite{metacriticrpg, amazonrpg, steamrpg}. And even more interestingly, 
most user reviews that explicitly mention the user interface are very happy 
with it, calling it ``clean'' and ``intuitive'' amongst other things.

The map editor is simple and intuitive to anyone who's ever used a bitmap 
editor of any kind. But the plethora of nested menus that needs to be endured 
to do anything else is unkind to the overall user experience. Add to that a 
poor overview, complex games quickly burdens the mind with the need to 
remember how your pieces all fit together, if they do so at all.

It may appear user reviews focus more on virtues of the map editor, whilst 
professional reviews can't get past the troubling interface for game objects 
in general.

\subsubsection{Takeaway points}

\begin{itemize}
\item What You See Is What You Get is well-liked amongst users.
\item Horribly complicated nested menu systems are, uh, well, not well-liked 
    amongst professional reviewers\ldots
\item Authoring games can be a successful way of learning.
\item Playing games can successfully motivate learners.
\end{itemize}
