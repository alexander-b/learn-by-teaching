\subsection{H5P}

H5P is a standard allowing for reusable, portable e-learning modules. It does so
by providing shims for different CMSs, a file format specification for
distribution, as well as standard \textit{content types} with built-in editors.

The value proposition of H5P is that it integrates nicely with existing
infrastructure like databases and view layers, enabling developers to focus on
developing their modules.\cite{h5pwhy} That is, it reduces the ceremony needed
to get going. Moreover, its portability provides a much-wanted ease of
deployment for site owners, as well as a bigger market for developers.

Five issues have been raised as part of the Learning by Teaching (LBT) project.
LBT seeks to provide a canvas which offers a visual way for users, not
developers, to creat their own e-learning modules. The issues raised are:

\begin{itemize}
\item Are the abstractions offered by H5P powerful enough?
\item Are the abstractions low- or high-level enough for our purposes?
\item Could LBT generate valid H5P modules?
\item Could LBT import (other) H5P modules?
\item Could LBT offer a new way of composing H5P?
\end{itemize}

This evaluation primarily seeks to (partially) answer most of these questions.
It does not in particular seek to provide concrete improvements to the H5P
specification, neither does it guarantee that H5P is (not) amenable to any
purpose.

\subsubsection{Are the abstractions offered by H5P powerful enough?}

At its very core, H5P offers a contract for libraries, which might function as
full-blown applications, stand-alone libraries, or APIs for any combination of
the above. 

This is done in part by making the libraries document their semantics
(`semantics.json`), that is the properties (data structure) that is used to set
up the application. Each property is described by its name, as well as its type.
Different content types have different semantic requirements: For example, the
`list` (ordered group) semantic type requires the name of a single entity in it
(`entity`). This information is used by the generic editor to allow users to
easily customise the library.

H5P also provides a container format that takes care of ceremony, like loading
the appropriate CSS and JS files, as well as providing useful metadata like the
version and name of the library. This format includes a JS class which libraries
extend to hook into H5P. Applications are required to implement an `attach`
method, which accepts the id of the container in which to attach the module.

Of particular interest for LBT, it provides a central event dispatcher
(`H5P.EventDispatcher`)\cite{h5pdispatch}, which is used to dispatch actions
within H5P. Thus, like Elm's ports or Facebook React's Flux
architecture/dispatcher, a synchronous interface for internal communication is
provided, which might prove useful for multi-component systems.

Beyond this, H5P leaves the library developers plenty of room to expand. The H5P
standard is pretty much "open world" and doesn't necessarily limit the structure
of functionality of the JavaScript libraries it wraps, enabling developers to
create arbitrarily complex software.

However, it should be noted that H5P is still dependent upon professionalt
developers, and doesn't provide primitives that necessary make any sense for
ordinary users. Therefore, the standard by itself doesn't preclude a more
user-oriented canvas system like LBT.

In conclusion, H5P doesn't itself provide the needed abstractions necessary, but
doesn't hinder them from being developed. LBT targets a different demographic.

\subsubsection{Are the abstractions low- or high-level enough for our purposes?}

LBT is still in its early stages. Thus, it is not known how low-level the
abstractions need to be to faciliate such a system. However, the problem domain
is vastly different: Yes, both LBT and H5P might rely on defining different data
types, and need to be able to express common scenarios within digital learning
situations. But whereas H5P seems to be primarily geared towards providing
consistent deployment stories for adminstrators and consistent semantics for
developers, LBT seeks to a provide a new type of user experience, namely
creating new e-modules. The latter is probably not general enough that it
warrants inclusion in the H5P standard itself, but enhances the experience of
e-learning production for a new demographic. 

In conclusion, this question does not seem to have a good answer for the time
being. Rather, it is sufficient to suggest that the abstractions for H5P and LBT
would be different. And because H5P allows developers great freedom in
developing their application, as long as they follow the contract, this
shouldn't prove to be a problem.

\subsubsection{Could LBT generate valid H5P modules?}

Yes. As already noted, H5P doesn't limit any system it wraps in its container
format. Therefore, any functioning JavaScript system could in principle be
embedded within a H5P module, provided it only operates on a single node (due to
the complexity and performance implications of coordinating DOM operations on
overlapping elements).

\subsubsection{Could LBT import (other) H5P modules?}

Yes, provided a shim is written that is able to parse the semantic definition of
the H5P module and use it in some meaningful way. In this case, LBT would do not
much more than the generic editor provided by H5P, but might provide a different
user story and target new demographics not yet reached by H5P. (Students or
junior high teachers aren't known for adminstrating installations of Wordpress
or Drupal, at least not widely).

\subsubsection{Could LBT offer a new way of composing H5P?}

H5P has no composition guarantees, and resorts to Design by Contract (DbC) for
composite systems.\cite{h5pdbc} This does nothing to ensure a uniform
communication standard, which ensures flexibility, but does not provide a
uniform interface. The specification itself admits that this is due to the lack
of interfaces in JavaScript. Thus, the contracts are more by politeness and not
by obligation, potentially creating weak implementations.

Therefore, there is obviously definitely value to be provided by a system
providing more stringent, verifiable contracts between components, for example
using a more powerful type system than the one found in JavaScript. This might
in turn result in more reusable and robust apps. Also, because H5P effectively
provides no composition contracts, this system might function as a sandbox and
provide useful feedback should future versions of H5P seek to provide more
formal requirements for composite modules.

\subsubsection{Minor issues raised}

\begin{itemize}
\item H5P provides no standard for sub-module inheritance/reuse of content, as the
system only has a notion of full H5P modules. How does two modules refer to
the same data?
\end{itemize}


