\subsection{Concept }
We propose a canvas system wherein the user may drag e-learning modules from a 
toolbox, and drop them onto the canvas. The modules may be empty or finished. 
If they are empty, then the authoring tool for said module opens. If it is not 
empty, then the user may further edit it if they wish.

To schedule data and control flow, the user drags arrows between the modules 
on the canvas. These associations may be customised. ``If the user finished 
[module a] with a score of more than 80\% correct, then direct them to [module 
b]; if the user did not, then direct them to [module c]''.

Compositions are modules themselves.

When a user wants to compose modules, users may search for which modules to 
use. We want to encourage reuse and remixing rather than novelties. 
Consequently we focus the search option before the ``author new module'' 
option. As another incentive, when a user authors a new module, we track 
reuses and remixes. Users may in a presentation of all their authored modules 
see how many reuses and remixes have occurred, and follow links to these. If 
they like a remix they may merge the changes between the remix and the 
original.

Compositions may be derived in whole, and further developed by another author. 
Users may merge modifications to these as well. Consider an English Language 
test for a school curriculum. A teacher at a different school (or the same 
school next year) may want to reuse this test since they have the same 
curriculum. But they might furthermore want to modify it. If the original 
author finds these modifications useful, then it may merge them to the 
original.

Avatars have been successfully used before to motivate 
children\cite{gossen2012search}. We propose the use of an avatar for 
explaining the interface and pointing out problems with the compositions, such 
as integrity issues, e.g.\ ``the composition never ends'', or continuity 
warnings, e.g.\ ``this module is visited two times'',. The avatar's appearance 
may be customised.

To encourage certain behaviour, the avatar may reward positive behaviour with 
a badge, e.g.\ ``200 remixes!'' --- in total, or ``remixed 5 times!'' --- for 
one specific module.

Punishing bad behaviour rarely works, and punishment is easily evaded. Instead 
we simply prohibit bad behaviour. Chatting is done via the avatars. To 
communicate with another user, you visit their profile and click on their 
avatar to bring up a dial menu for constructing a query like ``you should add 
[module] to your composition!'', or ``I like this module!''. This makes 
localisation much easier since these set sentences may be translated 
accurately. It also makes our system safe for 
children\cite{sadler2012virtual}. This is a recent trend in computer games 
such as Journey and Hearthstone for the same reasons. This eliminates hate 
speech\cite{hearthstone}, which is not acceptable with a child audience.

Users may, via their avatars as just described, encourage other users by 
telling them that they like their modules. There is no way to dislike a 
module, as that might be demotivating for children. Avatars as well as modules 
may be favoured. The user's favourite modules and avatars are more easily 
accessible than the rest.

Users sign up with a log-on id and password, and optionally an email address 
for recovering forgotten passwords. The users name their avatars which is then 
their screen names. The avatar has a given name and a surname. There may be 
duplicate names. People have duplicate names in real life too [citation 
needed], and society hasn't collapsed yet.

\subsection{Requirements}
We place high usability requirements on our canvas system in order to enable
school children to use it. Universal design is difficult with such a graphical 
system, but of the utmost importance. Compromises shall err on the side of 
caution and favour universal design.

Dealing with children has implications. Tap actions should allow some slack 
with bigger hit boxes than actual buttons and similar measurements. Splash 
screens are bad, mkay. Turns out kids don't have that great an attention-span. 
So get to the point quick as. These things are discussed in more detail in 
Section~\ref{principles}.

In order to encourage reuse and remixing, modules and assets are distributed 
under a free licence, CC-BY-SA\@. This also sets a positive example for school 
children to become good citizens of our society. Indeed so does the entire 
project by being licensed as AGPL\@\cite{educational}.

\subsection{Feature List}
\begin{itemize}
\item Canvas system
\item Toolbox
\item Search
\item Favourite modules
\item Favourite avatars
\item Reuse and remix tracking
\item Merging
\item Avatars
\end{itemize}

\subsection{Functional Details}

\subsection{Wireframes etc.}
