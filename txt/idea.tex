We suggest developing an engaging ``canvas''  for authoring e-learning 
modules, taking cues from 
\begin{CJK}{UTF8}{min}RPGツクール\end{CJK}\footnote{\url{http://www.rpgmakerweb.com/}}, 
Game Maker\footnote{\url{https://www.yoyogames.com/studio}}, and other similar 
software that succeed in take advantage of an intuitive interface with a 
capacious featureset. By ``canvas'' (or ``scene graph''), we mean an 
easy-to-learn system of widgets and composite stores (with rich content), and 
flows connecting them together to a cohesive system. 

An example of such a system might be a quiz leveraging media like GNU 
Mediagoblin, Nasjonal Digital Læringsarena\footnote{\url{http://ndla.no/}}, 
Youtube, NRK, and Twitter to showcase the issue at hand, before testing the 
user's comprehension at the end. Thus, the student-come-software-developer is 
able to prove their comprehension of multimedia (as mandated by the 
curriculum), and other students allowed to take part in the learning 
experience. Incentives for doing so might be given by the teacher, or by 
arranging competitions that invite the students to share their best ideas.

Taking cue from Minecraft, we also propose another viable incentive: The 
student is provided with ready-made canvases with actors and stores, and 
encouraged to make the application perform certain actions (``gamification''). 
This is akin to the redstone system found in Minecraft, where even young users 
are able to construct discrete logic circuits providing useful functionality 
like opening doors, or switching on lights\cite{brand2013crafting}.

Through use of our canvas, we eliminate the BOP by liberating and empowering 
it to make its own user experience. As an added bonus, our canvas may spark 
some latent creative souls, or inspire technological awareness and interest. 
In our increasingly computerised society, this is in itself a noble cause. 

Thus, this canvas might prove to be a force for bridging the gap between just 
being a computer user, and having a promising future career in computer 
software. While Computing At 
School\footnote{\url{http://www.computingatschool.org.uk/}} have had great 
success in the UK, there is as of today no readily-available path for 
acquiring the advanced knowledge needed to develop modern systems given the 
current education system in Norway. Grassroots organisations such as Lær Kidsa 
Koding\footnote{\url{http://www.kidsakoder.no/}} are doing good work, but have 
yet to strongly influence the education system. If our canvas is picked up by 
prominent e-learning providers like Nasjonal digital læringsarena we can 
liberate and empower users through direct action, circumventing bureaucracy.

In addition to users teaching themselves technology, they also teach 
themselves the curriculum more effectively. The student becomes the teacher, 
and we achieve learning by teaching, an often sought-after method of 
reinforced learning. By only being e-learning module \emph{users}, students 
are limited to learning through observation, experimentation, and (to some 
extent) mistakes. Learning through teaching offers advantages not possible to 
fully realise through either of these; advantages that won't manifest if the 
student is exclusively relying on an external 
teacher\cite{cortese2005learning}.

Our canvas makes composing e-learning modules intuitive for nen-technical 
users and at the same time powerful enough to entice power users and 
established e-learning module authors. Consequently, the target demographic of 
our canvas is not limited to the BOP, but extends to include current 
e-learning module authors.

To further underline our empowering of the BOP, we make sharing of works very 
simple, and make it equally simple to author derivatives (forks) and 
meta-works (collections). Incentive for sharing your work, and improving or 
remixing the work of others is provided through gamification of the canvas. As 
an example, there may be a reward system for users whose modules are often 
remixed, and for users who make an improvement to a module that then gets 
integrated back into the original module itself.

With gamification, we can also help ensure high quality modules. With a rating 
system and achievements for obtaining a high rating, we realise a 
self-regulating community.

Initially we aim to support authoring H5P modules, focussing our development 
at H5P integration. But with a good modular design we can extend our canvas to 
support other standards as well in the future.
